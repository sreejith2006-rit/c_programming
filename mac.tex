\documentclass[11pt,a4paper]{article}
\usepackage{fullpage}
\usepackage{graphicx}
\usepackage{amssymb,amsmath,amsfonts}
\title{\textbf{Mathematics Section}}
\author{}
\date{}
\begin{document}
\maketitle
\section{{Introduction}}
\begin{Large}
    \begin{center}
        \textbf{Macleurin Series : A Comprehensive Overview}
    \end{center}
\end{Large}
\begin{large}
    

\begin{flushleft}
The \textit{Maclaurin series} is a special case of the \textit{Taylor series}, which is a representation of a function as an infinite sum of terms derived from the values of its derivatives at a single point.The Maclaurin series specifically centers its expansion at $x=0$,
 making it an essential tool for approximating functions that are difficult to work with directly. It is named after the Scottish mathematician Colin Maclaurin, who first introduced it.
\end{flushleft}
\begin{center}
\includegraphics[scale=0.6]{image.png}\\
\textbf{Colin Maclaurin}
\end{center}
\end{large}
\textbf{General Formula for the Maclaurin Series:}\\
The Maclaurin series expresses a function $
f(x)$ as an infinite sum of terms. The general formula is given by:$$f(x)=f(0)+f'(0).\frac{x}{1!}+f''(0).\frac{x^2}{2!}+f'''(0).\frac{x^3}{3!}+\dots$$
Or, more compactly:
$$f(x)=\sum_{n=1}^{\infty}
\frac{f^{(n)}(0)}{n!}.x^n$$
Where:
\begin{itemize}
    \item $f^{(n)}(0)$ represents the \textit{n}-th derivative of $f(x)$ evaluated at $x=0$
    \item $n!$ denotes the factotial of $n$, and $x^{(n)}$ is the $n$-th power of $x$
\end{itemize}
\textbf{Key Concepts in the Maclaurin Series}
\begin{enumerate}
\item Function Derivatives: The Maclaurin series relies heavily on the derivatives of the function being expanded. he first term of the series is simply the function's value at $x=0$ ,the second term is the first derivative evaluated at 0, and so on. Each term involves higher-order derivatives of the function.
\item Factorials: The terms in the series involve factorials, which grow rapidly. Factorials appear in the denominator, which causes the higher-order terms to decrease quickly in magnitude as $n$ increases.
\item Convergence: The Maclaurin series approximates the function as a sum of terms. For many functions, especially well-behaved ones like polynomials, exponentials, and trigonometric functions, the series converges to the function's value as more terms are included. However, the series may not converge for all functions or for all values of $x$.
\item Applications: The Maclaurin series is widely used to approximate functions that do not have simple closed forms or are difficult to evaluate directly. It is particularly useful in fields like physics, engineering, and computer science, where approximations to trigonometric, exponential, logarithmic, and other complex functions are needed. In computational methods, truncating the series after a few terms often provides sufficient accuracy.
\end{enumerate}
\section{Derivation of $\sin(x)$
and $\cos(x)$ using maclaurin series}
\subsection{Steps to derive the Maclaurin series for $\sin(x)$:}
\begin{enumerate}
\item Function: The function we are working with is $f(x)=\sin(x)$
\item Find the derivatives of $f(x)$.We calculate the derivatives of $\sin(x)$ at $x=0$:
\begin{itemize}
    \item $f(0)=\sin(0)=0$
    \item $f'(0)=\cos(0)=1$
    \item $f''(0)=-\sin(0)=0$
    \item $f'''(0)=-\cos(0)=-1$
    \item $f''''(0)=\sin(0)=0$
\end{itemize}
And so on. The derivatives of $\sin(x)$ repeat in a cycle every four terms:\\
$\sin(x),\cos(x),-\sin(x),-\cos(x)$
\item Substitute into the Maclaurin series formula: The Maclaurin series is:$$f(x)=\sum_{n=1}^{\infty}
\frac{f^{(n)}(0)}{n!}.x^n$$
Substituting the derivatives of $\sin(x)$:
\begin{align*}
    \sin(x)&=\sin(0)+f'(0).\frac{x}{1!}+f''(0).\frac{x^2}{2!}+f'''(0).\frac{x^3}{3!}+f''''(0)\frac{x^4}{4!}+\dots\\
    &=0+\frac{1}{1!}x+\frac{0}{2!}x^2+\frac{-1}{3!}x^3+\frac{0}{4!}x^4+\frac{1}{5!}x^5+\dots\\
    &=x-\frac{x^3}{3!}+\frac{x^5}{5!}-\frac{x^7}{7!}+\frac{x^9}{9!}-\dots\\
\end{align*}
\item\textbf{Final Maclaurin series for $\sin(x)$:}
$$\boxed{\sin(x)=x-\frac{x^3}{3!}+\frac{x^5}{5!}-\frac{x^7}{7!}+\frac{x^9}{9!}-\dots\\}$$
$$or$$
$$\boxed{\sin(x)=\sum_{n=0}^{\infty}(-1)^n\frac{x^{2n+1}}{(2n+1)!}}$$
\end{enumerate}
\subsection{Steps to derive the Maclaurin series for $\cos(x)$}
\begin{enumerate}
\item Function: The function we are working with is $f(x)=\cos(x)$
\item Find the derivatives of $f(x)$.We calculate the derivatives of $\cos(x)$ at $x=0$:
\begin{itemize}
    \item $f(0)=\cos(0)=1$
    \item $f'(0)=-\sin(0)=0$
    \item $f''(0)=-\cos(0)=-1$
    \item $f'''(0)=\sin(0)=0$
    \item $f''''(0)=\cos(0)=1$
\end{itemize}
And so on. The derivatives of $\cos(x)$ repeat in a cycle every four terms:\\
$\cos(x),-\sin(x),-\cos(x),\sin(x)$
\item Substitute into the Maclaurin series formula: The Maclaurin series is:$$f(x)=\sum_{n=1}^{\infty}
\frac{f^{(n)}(0)}{n!}.x^n$$
Substituting the derivatives of $\cos(x)$:
\begin{align*}
    \cos(x)&=\cos(0)+f'(0).\frac{x}{1!}+f''(0).\frac{x^2}{2!}+f'''(0).\frac{x^3}{3!}+f''''(0)\frac{x^4}{4!}+\dots\\
    &=1+\frac{0}{1!}x+\frac{-1}{2!}x^2+\frac{0}{3!}x^3+\frac{1}{4!}x^4+\frac{0}{5!}x^5+\dots\\
    &=1-\frac{x^2}{2!}+\frac{x^4}{4!}-\frac{x^6}{6!}+\frac{x^8}{8!}-\dots\\
\end{align*}
\item\textbf{Final Maclaurin series for $\cos(x)$:}
$$\boxed{\cos(x)=1-\frac{x^2}{2!}+\frac{x^4}{4!}-\frac{x^6}{6!}+\frac{x^8}{8!}-\dots\\}$$
$$or$$
$$\boxed{\cos(x)=\sum_{n=0}^{\infty}(-1)^n\frac{x^{2n}}{(2n)!}}$$
\pagebreak
\section{Practical Applications :}
In practice, the Maclaurin series can be truncated after a certain number of terms to provide an approximation that is sufficiently accurate for most applications.
\subsection{Some other important expansions using Maclaurin Series}
\begin{Large}
    \begin{itemize}
    \item $e^x = 1 + x + \frac{x^2}{2!} + \frac{x^3}{3!} + \frac{x^4}{4!} + \dots$
    \item $\ln(1 + x) = x - \frac{x^2}{2} + \frac{x^3}{3} - \frac{x^4}{4} + \dots$ $\forall$ \ \text{  $|x|<1$}
    \item $\tan(x) = x + \frac{x^3}{3} + \frac{2x^5}{15} + \frac{17x^7}{315} + \dots$
    \item $\sinh(x) = x + \frac{x^3}{3!} + \frac{x^5}{5!} + \frac{x^7}{7!} + \dots$
    \item $\cosh(x) = 1 + \frac{x^2}{2!} + \frac{x^4}{4!} + \frac{x^6}{6!} + \dots$
    \item $\tan^{-1}(x) = x - \frac{x^3}{3} + \frac{x^5}{5} - \frac{x^7}{7} + \dots$    $\forall$ \ \text{  $|x|\leq1$}
     \end{itemize}
\end{Large}




\end{enumerate}

\end{document}